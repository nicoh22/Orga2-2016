Las tecnologias actuales de video en tiempo real y filtros de imágen son posibles desde hace unos años gracias a la eficiencia del código usado en sus implementaciones. Sería imposible una implementación de un reproductor streaming puramente en un lenguaje de programación de uso común como C, debido a la enorme cantidad de computos requerida.\\

Para solucionar este inconveniente se requería el procesamiento de información en paralelo y esto fue posible gracias a las instrucciones SIMD (single instruction multiple data) diseñadas y desarrolladas por intel, que son un subconjunto de las instrucciones bien conocidas assembler que utiliza esta arquitectura, las cuales permiten operar con varios datos a la vez, facilitando el calculo de pixeles a la hora de procesar una imagen de video y realizar operaciones de punto flotante, muy importantes en la creación de gráficos por computadora para el calculo de fisicas, iluminación, etc.\\
Tambien tiene aplicación en el procesamiento de filtros de imágen, obteniendo con los mismos, resultados notables frente a implementaciones con otros lenguajes usuales como el mencionado C.

\subsection{Motivaciones}

En el siguiente informe haremos uso del set instrucciones de SIMD para implementar tres filtros, a saber: Sepia, Cropflip y Low Dynamic Range, los cuales serán explicados y desarrollados en el mismo. Propondremos además una variante de la implementación de cada filtro en lenguaje C.\\

Luego abordaremos sus caracteristicas particulares y propondremos casos de estudio en base a los mismos para intentar responder a preguntas como: Qué implementación es mejor?. Primero se analizarán las metodologias para llevarlos a cabo y luego se darán a conocer los resultados obtenidos. Por último, en base a lo mostrado, obtendremos las conclusiones pertinentes.