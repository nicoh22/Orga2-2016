Como pudimos comprobar, la importancia de SIMD para resolver este tipo de problematicas es muy significativa. 
Pudo observarse que los tiempos para un algoritmo en un lenguaje convencional como $C$ son muy elevados llegando a requerir aproximadamente 4 veces más de tiempo para realizar la misma tarea que en lenguaje $assembler$ $($es decir que con assembler se obtiene una eficiencia del 400$ \% )$\\

Pudimos observar detalles particulares a cada implementación. En especial, obtuvimos que para el filtro $ldr$ no era conveniente operar con enteros, pero esto parece estar más asociado al tipo de implementación y los cambios requeridos para lograr divisiones con enteros. Por ejemplo, si se hubiese realizado este mismo test sobre $sepia$ muy probablemente hubiese arrojado otros resultados debido a que operariamos con enteros sin perder paralelismo.\\

Otro resultado que pudimos observar, tambien con el filtro $ldr$, es que los accesos a memoria no suponen una perdida de rendimiento, dado que traer de a cuatro o traer de a un pixel es igual, debido a que la caché utiliza principio de vecinidad espacial y al leer un pixel la cache trae varios pixeles contiguos. La diferencia radica principalmente en poder realizar operaciones de forma paralela, siendo nuevamente este el punto fuerte de SIMD. \\


